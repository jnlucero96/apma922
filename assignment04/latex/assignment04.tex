\documentclass[12pt]{article}

% Document Formatting Packages
\usepackage{geometry}
\geometry{letterpaper}
\usepackage[usenames,dvipsnames,svgnames,table]{xcolor}

% Document Navigation Packages
\usepackage[parfill]{parskip}
\usepackage{enumitem}

% Math Typesetting Tools
\usepackage{amssymb}
\usepackage{amsmath,mathtools}
\usepackage{framed}
\usepackage{bm} % boldface greek symbols
\usepackage{nicefrac}
\usepackage{upgreek}

% Hyperref
\usepackage[colorlinks=true,linkcolor=blue,citecolor=red]{hyperref}

% Color Text Tools
\newcommand{\red}[1]{\textcolor{Red}{#1}}
\newcommand{\blue}[1]{\textcolor{Blue}{#1}}
\newcommand{\green}[1]{\textcolor{Green}{#1}}

% Chemistry Typesetting Tools
\usepackage{expl3}
\usepackage{calc}
\usepackage{mhchem}

% Physics Typesetting Tools
\usepackage{physics}

% Inserting Figures
\usepackage{graphicx}
\graphicspath{ {images/} }
\usepackage[caption=false]{subfig}
\usepackage[section]{placeins}

% Miscellaneous Symbol Packages
\usepackage{textcomp}
\usepackage{siunitx}
\usepackage{gensymb}

% Set Document Dimensions
\oddsidemargin = 0in
\topmargin = 0in
\headheight=0pt
\headsep = 0pt
\textheight = 9in
\textwidth = 6.5in
\marginparsep = 0in
\marginparwidth = 0in
\footskip = 18pt
\parindent=0pt
\parskip=0pt

% Symbol shortcuts
\newcommand{\kT}{k_{\mathrm{B}}T}
\newcommand{\xbar}{\bar{x}}
\newcommand{\ybar}{\bar{y}}
\newcommand{\bO}{\mathcal{O}}
\newcommand{\vhat}{\hat{v}}
\newcommand{\dfte}{e^{-ikx_{j}}}
\newcommand{\idfte}{e^{ikx_{j}}}

\allowdisplaybreaks

%\newtheorem{theorem}{Theorem}

% Title
\title{APMA 922: Homework Set 04}
\author{Joseph Lucero}
\date{\today}

\begin{document}
\maketitle

% =================================================================================
\section*{Problem 1}

\subsection*{Part A}

In order to apply TChebyshev spectral methods to solve BVPs we must scale and translate the domain variable $x \in [0,1]$ of the given BVP into the variable $z\in [-1,1]$. To do this we let $$z = 2x-1,$$ and thus have that $$x = \dfrac{z+1}{2}.$$ So the variable $z\in [-1,1]$ and thus we can apply TChebyshev spectral methods on this variable. With this transformation the differentials are also transformed. In particular we have $$\dd{x} = \dfrac{1}{2}\dd{z},$$ which then transforms the definition of the derivative $$ \dv{x}\to 2\dv{z},$$ and similarly for the second derivative $$\dv[2]{x}\to 4\dv[2]{z}.$$ Thus the differential equation in variable $x$ is recast into the differential equation in variable $z$, 
\begin{align}
	\left[4\varepsilon\dv[2]{z} + 2\dv{z}\right]y(z) = -\exp\left(\dfrac{1-z}{2}\right),\quad -1 < z < 1,\quad y(-1) = y(1) = 0.
\end{align}

Figure~\ref{fig:q1a_figure} shows, in the left subplot, the solution acquired from the spectral method in the black curve. As can be seen the computed solution agrees very well with the exact solution that is shown with the red curve. The right subplot of Fig.~\ref{fig:q1a_figure} we observe that for $\varepsilon=1$, for $N=16$, the error of the solution is already as small as it can be. We also note that the $N$, for which the error is minimum, is increased the smaller $\varepsilon$ is. This, as we saw before, is because a boundary layer appears as $\varepsilon\to 0$ and thus requires more points to resolve. We observe that it requires about an accuracy of $10^{-5}$ is achieved for $N=5$.

\begin{figure}[!h]
	\centering
	\subfloat{\includegraphics[clip, scale=0.3]{q1a_soln_figure.pdf}}
	\subfloat{\includegraphics[clip, scale=0.3]{q1a_err_figure.pdf}}
	\caption{[Left] Exact (red line) and computed (black points) solution as a function of $x$ for $N=24$ and $\varepsilon=1$. [Right] Infinity-norm error of the computed solution for $\varepsilon=1$ (blue line) and $\varepsilon=0.01$ (green line) as a function of the number of intervals $N$.}
	\label{fig:q1a_figure}
\end{figure}

\subsection*{Part B}

As above, we rescale the domain variable $x\in [-2, 1]$ to a variable $z\in[-1,1]$. To do this we let $$z = \dfrac{2x+1}{3},$$ and thus have that $$x=\dfrac{3z-1}{2}.$$ So the variable $$z\in[-1,1]$$ and we can apply TChebyshev spectral methods on this variable. With this transformation the differentials are also transformed. In particular we have $$\dd{x} = \dfrac{3}{2}\dd{z},$$ which then tranforms the definition of the derivative $$\dv{x}\to \dfrac{2}{3}\dd{z},$$ and similarly for the second derivative $$\dv[2]{x}\to \dfrac{4}{9}\dv[2]{z}.$$ Thus the differential equation in variable $x$ is recast into the differential equation in variable $z$,
\begin{align}
	\left[\dfrac{4}{9}\dv[2]{u}{z} - 4(3z-1)\right]u(z) = -(3z-1)^{2},\quad -1 < z < 1,\quad u(-1) = 0,\ u^{\prime}(1) = 2. 
\end{align}

\begin{figure}[!h]
	\centering
	\subfloat{\includegraphics[clip, scale=0.3]{q1b_soln_figure.pdf}}
	\subfloat{\includegraphics[clip, scale=0.3]{q1b_err_figure.pdf}}
	\caption{[Left] Computed solution as a function of $x$. [Right] Infinity-norm error of the computed solution as a function of the number of intervals $N$.}
	\label{fig:q1b_figure}
\end{figure}

Figure~\ref{fig:q1b_figure} shows, in the left subplot, the solution acquired from the spectral method in the black curve. The right subplot of Fig.~\ref{fig:q1b_figure} we observe that, as we expect, we attain spectral accuracy for this method. To get an accuracy of $10^{-4}$ we observe that we need $N=10$ number of intervals.

% =================================================================================
\section*{Problem 2}

\subsection*{Part A}

Using the definition of the differentiation matrix given that we derived in the last assignment,
\begin{align}
    D_{ij} = 
    \begin{cases}
        \dfrac{a_{i}}{a_{j}(x_{i}-x_{j})},&\quad i\neq j\\
        \sum\limits_{k=0,k\neq j}^{N}(x_{j}-x_{k}), &\quad i=j.
    \end{cases}    
\end{align}
we implemented this differentiation matrix in \verb|A4Q2.ipynb|. Using the TChebyshev extreme points $\{x_{j}\}$ in the definition above
returns the TChebyshev differentiation matrix as we expect. Demonstration of this test in one of the cells in the notebook.

\subsection*{Part B}

This is implemented in \verb|A4Q2.ipynb|. Since the algorithm in \verb|gauss.m| yields the $N$ zeros of the Legendre polynomial, to generate our grid in the interval [-1,1], the interior points are generated from $N-1$ zeros of the Legendre polynomial. The endpoints 
constitute the remaining two points that bring the total number of points in the grid to $N+1$, consistent with the output of the \verb|cheb.m| algorithm. 

\subsection*{Part C}

Figure~\ref{fig:tchebyshev_err} is a recreation of Output 11 from \textbf{[Tr]}. Here the function $u(x)$ is evaluated on the TChebyshev grid and the TChebyshev differentiation matrix is used to compute the spectral derivative. Figure~\ref{fig:legendre_err} is the recreation of Output 11 except using Legendre points to evaluate the function $u(x)$ and the Legendre differentiation matrix to compute the spectral derivative. To get the Legendre points I modified the code I acquired $N-1$ points from \verb|gauss.m| and then manually added the left and right hand end points of $-1$ and $1$, respectively, for a total of $N+1$ points.

\begin{figure}[!h]
	\centering
	\subfloat{\includegraphics[clip, scale=0.5]{tchebyshev_err_figure.pdf}}
	\caption{[Left column] Function $u(x)$ evaluated on the TChebyshev grid for $N=10$ (top left subplot) and $N=20$ (bottom left subplot) as a function of $x$. Associated infinity-norm error of the spectral derivative $u^{\prime}(x)$ for $N=10$ (top right subplot) and $N=20$ (bottom right subplot) as a function of $x$.}
	\label{fig:tchebyshev_err}
\end{figure}

\begin{figure}[!h]
	\centering	
	\subfloat{\includegraphics[clip, scale=0.5]{legendre_err_figure.pdf}}
	\caption{[Left column] Function $u(x)$ evaluated on the Legendre grid for $N=10$ (top left subplot) and $N=20$ (bottom left subplot) as a function of $x$. Associated infinity-norm error of the spectral derivative $u^{\prime}(x)$ for $N=10$ (top right subplot) and $N=20$ (bottom right subplot) as a function of $x$.}
	\label{fig:legendre_err}
\end{figure}

\subsection*{Part D}

Figure~\ref{fig:q2d_figure} shows the computed solution for the two BVPs that we are asked to solve. In the left subplot is the computed solution for the differential equation $u^{\prime\prime}(x) = e^{4x}$ acquired from utilizing the TChebyshev (red solid line) and the Legendre (black dashed line) spectral differentiation matrix. Since for this BVP $x\in [-1,1]$, no rescaling of the domain was necessary. 

In the right subplot the computed solution for the differential equation $u^{\prime\prime}(x) - 8xu = -4x^{2}$ acquired from utilizing the TChebyshev (red solid line) and the Legendre (black dashed line) spectral differentiation matrix. The scaling of the domain used in Question 1 Part B was used once again to solve the BVP on the domain $z\in [-1,1]$. 

\begin{figure}[!h]
	\centering
	\includegraphics[clip, scale=0.3]{q2d_soln_figure1.pdf}
	\includegraphics[clip, scale=0.3]{q2d_soln_figure2.pdf}
	\caption{[Left] Computed solution, $U(x)$, of the differential equation $u^{\prime\prime}(x) = e^{4x}$ as a function of $x$. [Right] Computed solution, $U(x)$, of the differential equation $u^{\prime\prime}(x) - 8xu = -4x^{2}$ as a function of $x$. Red solid lines are solution computed using TChebyshev differentiation matrices. Black dashed lines are solution computed using Legendre differentiation matrices.}
    \label{fig:q2d_figure}
\end{figure}

% =================================================================================
\section*{Problem 3}

\subsection*{Part A}
As in the discussion we have that the forward Euler method for this problem, with constant step size $k$ produces iterates of the form $U^{n} = (1+k\lambda)^{n}$. As in the problem discussion we have use $N = 200$ with steps of size $k=0.01$. This yields the computed solution 
\begin{subequations}
    \begin{align}
        U^{N} &= (1+(0.01)10)^{200}\\
        &= (1.1)^{200}\\
        &\approx \num{1.90e8}.
    \end{align}
\end{subequations}
The true solution is $u(T) = e^{20} \approx \num{4.85e8}$. Thus we observe that for this case the error bound becomes a more reasonable estimate of the error. The Forward-Euler time stepping scheme is implemented in \verb|A4Q3.ipynb|.

\subsection*{Part B}
The general explicit one-step method takes the form 
\begin{align}
    U^{n+1} = U^{n} + k\Psi(U^{n},t_{n},k).\label{eq:one_step_form}.
\end{align}
In this case, since $\Psi$ is linear in $u$, we have that 
\begin{align}
    \Psi(U^{n},t_{n},k) - \Psi(U^{n},t_{n},k) = \lambda \left[U^{n} - u(t_{n})\right] = \lambda E^{n}.\label{eq:linear_lipschitz}
\end{align}
Given this we can then write the global error as
\begin{subequations}
    \begin{align}
        E^{n+1} &= E^{n} + k\left[\Psi(U^{n},t_{n},k) - \Psi(U^{n},t_{n},k)\right] - k\uptau^{n}\\
        E^{n+1} &= (1+k\lambda)E^{n} - k\uptau^{n}.
    \end{align}
\end{subequations}
By induction we have that 
\begin{align}
    E^{n} &= (1+k\lambda)^{n}E^{0} - k\sum\limits_{m=1}^{n}(1+k\lambda)^{n-m}\uptau^{m-1}
\end{align}
Taking absolute values, applying triangle inequality, we have
\begin{align}
    |E^{n}| &\le |1+k\lambda|^{n}|E^{0}| + k \sum_{m=1}^{n} |1+k\lambda|^{n-m}|\uptau^{m-1}|
\end{align}
We thus identify the Lipshitz constant of $\Psi$ as $M = |1+k\lambda|$. Using that $M \le 1$,
\begin{subequations}
    \begin{align}
        |E^{n}| &\le |E^{0}| + k \sum_{m=1}^{n}|\uptau^{m-1}|\\
        &\le |E^{0}| + t\norm{\uptau}_{\infty},
    \end{align}
\end{subequations}
where we have that $N = T/k$ is the number of time steps needed to reach time $T$ and where we upper bounded the sum of the truncation errors with the maximum truncation error $$\norm{\uptau}_{\infty} = \max_{0\le n\le N-1}{|\uptau^{n}|},$$ as well as using that $t = nk$ from the definition of the total number of time steps. Now for $t \le T$ we can write
\begin{align}
    |E^{n}| &\le |E^{0}| + T\norm{\uptau}_{\infty},
\end{align}
and stipulating the $|E^{0}| = 0$ then we have 
\begin{align}
|E^{n}| &\le T\norm{\uptau}_{\infty},
\end{align}
as desired.

\subsection*{Part C}
We show above that we can take $M = |1+k\lambda|$, and so if we now apply it to the given case we obtain the bound
\begin{align}
    |E^{n}| \le 2\cdot \dfrac{1}{2}k\norm{u^{\prime\prime}}_{\infty} = 100
\end{align}
which indeed is much better than the one obtained previously as it lacks the exponential term. 

\subsection*{Part D}

\subsubsection*{i)}
Following the hint, 
\begin{subequations}
    \begin{align}
        |u-u^{*}| &= |\Phi(v) - \Phi(v^{*})|\\
        &= |g^{-1}(v) - g^{-1}(v^{*})|\\
        &= |g^{-1}\left(\left[u-kf(u)\right]\right) - g^{-1}\left(\left[u^{*}-kf(u^{*})\right]\right)|
    \end{align}
\end{subequations}

\subsubsection*{ii)}

The Backward-Euler discretization of a initial value problem $u^{\prime} = f(u, t)$ is given by
\begin{align}
    \dfrac{U^{n+1}-U_{n}}{k} = f(U^{n+1}),
\end{align}
and thus can be recast into the form 
\begin{align}
    U^{n} = U^{n+1} - kf(U^{n+1}).
\end{align}
Then, if we define as in the discussion of the previous part $g(U^{n+1}) := U^{n+1} - kf(U^{n+1})$ so we have
\begin{align}
    U^{n} = g(U^{n+1})
\end{align}
and subsequently, defining $\Phi(U^{n}) := g^{-1}(U^{n})$, we see that if we apply this function $\Phi$ on both sides of the equation we have that
\begin{subequations}
    \begin{align}
        \Phi(U^{n}) &= \Phi(g(U^{n+1})) \\
        &= U^{n+1}.
    \end{align}
    \label{eq:backward_euler_form}
\end{subequations}
Then we see the backward Euler scheme takes the form desired. Since from i) we have that $\Phi$ is Lipschitz continous with Lipschitz constant $\mathcal{L} \le 1$ and we have from Part B that equations of the form~\eqref{eq:backward_euler_form} can have the error be bounded uniformly of the step size thus the backward Euler converges. 



% =================================================================================

\section*{Problem 4}

\subsection*{Part A}

One way to derive the Adams family of methods is by writing 
\begin{align}
	u(t_{n+r}) = u(t_{n+r-1}) + \int_{t_{n+r-1}}^{t_{n+r}}\dd{s} f(u(s))
\end{align}
and using a quadrature rule to approximate
\begin{align}
	\int_{t_{n+r-1}}^{t_{n+r}}\dd{t}f(u(t)) \approx k\sum_{j=1}^{r-1}\beta_{j}f(u(t_{n+j})).
\end{align}
This quadrature rule can be derived by interpolating $f(u(t))$ by a polynomial $p(t)$ of degree $r-1$ at the points $t_{n},t_{n+1},\dots,t_{n+r-1}$ and then integrating the interpolating polynomial. In this case we are interested in interpolating $f(u(t))$ through the points $t-k,t,t+k$. 

The interpolating polynomial in the Lagrange basis is given by, 
\begin{align}
	p_{2}(t) = L(t) = \sum_{j=0}^{k}f^{j}l_{j}(t).
\end{align}
with basis functions,
\begin{align}
	l_{j}(t) = \prod_{\substack{0\le m\le k \\ m\neq j}}\dfrac{t-t_{m}}{t_{j}-t_{m}}.
\end{align}
The interpolant is then
\begin{align}
	p_{2}(t) &= f^{n-1}\left(\dfrac{t-t_{n}}{t_{n-1}-t_{n}}\right)\left(\dfrac{t-t_{n+1}}{t_{n-1}-t_{n+1}}\right) +  f^{n}\left(\dfrac{t-t_{n-1}}{t_{n}-t_{n-1}}\right)\left(\dfrac{t-t_{n+1}}{t_{n}-t_{n+1}}\right) \nonumber\\
	&\hspace{8cm}+ 
	f^{n+1}\left(\dfrac{t-t_{n-1}}{t_{n+1}-t_{n-1}}\right)\left(\dfrac{t-t_{n}}{t_{n+1}-t_{n}}\right)
\end{align}
Using that $t_{n-1}=t_{n}-k,t_{n+1}=t_{n}+k$, we have
\begin{align}
	p_{2}(t) &= \frac{2f^{n}(k+t-t_{n}) (k-t+t_{n})+(t-t_{n}) (f^{n+1}(k+t-t_{n})-f^{n-1}(k-t+t_{n}))}{2 k^2},
\end{align}
and integrating from time step $t_{n}$ to $t_{n}+k$ we then have
\begin{subequations}
	\begin{align}
		\hspace{-1cm}\int_{t_{n+1}}^{t_{n+2}}\dd{t}p_{2}(t) &= \int_{t_{n}+k}^{t_{n}+2k}\dd{t}p_{2}(t)\\
		&= \frac{(-4f_{n}+2f_{n-1}+2f_{n+1})}{12k^2}t^3 +\frac{(12f_{n}
		t_{n}-3f_{n-1}k-6f_{n-1}t_{n}+3f_{n+1}k-6f_{n+1}t_{n})}{12 k^2}t^2\nonumber \\ 
		&\hspace{4cm}\left.+\frac{\left(12f_{n}k^2-12f_{n}t_{n}^2+6
		f_{n-1} k t_{n}+6 f_{n-1} t_{n}^2-6f_{n+1}k t_{n}+6
		f_{n+1}t_{n}^2\right)}{12 k^2}t\right|_{t_{n}+k}^{t_{n}+2k}\\
		&= \dfrac{k}{12}\left(5f^{n-1}-16f^{n}+23f^{n+1}\right)
	\end{align}
\end{subequations}
And so we finally have the three-step Adams-Brashforth scheme
\begin{subequations}
	\begin{align}
		U^{n+2} &= U^{n+1} + \int_{t_{n+1}}^{t_{n+2}}\dd{t}p_{2}(t)\\
		&= U^{n+2} + \dfrac{k}{12}\left(5f^{n-1}-16f^{n}+23f^{n+1}\right).
	\end{align}
\end{subequations}
We can get it into the form given in the question by simply shifting the index by 1, 
\begin{align}
	U^{n+3} &= U^{n+2} + \dfrac{k}{12}\left(5f^{n}-16f^{n+1}+23f^{n+2}\right).
\end{align}




\subsection*{Part B}

\subsection*{Part C}

\subsection*{Part D}

\subsection*{Part E}

% =================================================================================

\section*{Problem 5}

\subsection*{Part A}

\subsection*{Part B}

\subsection*{Part C}

\subsection*{Part D}

\subsubsection*{i)}

\subsubsection*{ii)}

\subsubsection*{iii)}

\end{document}